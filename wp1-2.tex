\subworkpkg{1.2}

%% Tasks

%% A. Familiarize with the celestial body object of the mission (Moon,
%% Mars or Europa) and collect all the relevant information on it
%% (e.g., gravitation coefficient, radius, atmospheric density,
%% orbital period, eclipse periods, solar intensity received, albedo,
%% etc.).

%% B. Search for at least 5 existing (or planned) spacecrafts to be
%% used as a reference for your design, i.e.  for which the S/C
%% specifications and characteristics are similar to the ones expected
%% in your mission.  Collect information on the design of the
%% identified reference spacecrafts (e.g., mass and power budgets,
%% size, structure configuration, payload characteristics, orbital
%% parameters, etc.).

%% C. List a detailed set of requirements for your design, taking into
%% account the top-level mission requirements (from the project
%% description provided in Section 3) and the characteristics of the
%% identified reference spacecrafts. Remember: the requirements shall
%% be SMART (Specific, Measurable, Achievable, Realistic,
%% Time-bound)!! Whereas possible express them as numbers, preferably
%% in the form of a range of values.

%% D. From the complete list of requirements, identify at least 10
%% driving requirements for your design, i.e.  requirements that will
%% play a major role in the design process. Provide an adequate
%% justification for your choice.

%% Deliverables

%% D1.2.1. Table including all the relevant data on the celestial body
%% object of the mission.

\deliverable{1.2.1}

Table including all the relevant data on the celestial body object of
the mission.  Data is from
\cite{nasaeuropa,Jupiter,Europa,Europa2,Europa3}.

\begin{itemize}
\item{Orbit Size Around Jupiter (semi-major axis):} \SI{671,100}{km}
\item{Periapsis (closest):} \SI{664,792}{km}
\item{Apoapsis (farthest):} \SI{677,408}{km}
\item{Sidereal Orbit Period (Length of Year):} 3.551181041 Earth days
\item{Orbit Circumference:} \SI{4216552.51}{km}
\item{Average Orbit Velocity:} \SI{49476.1}{km/h}
\item{Orbit Eccentricity:} 0.0094
\item{Orbit Inclination:} \ang{0.466}
\item{Mean Radius:} \SI{1560.8}{km}
\item{Equatorial Circumference:} \SI{9806.8}{km}
\item{Volume:} \SI{15926867918}{km^3}
\item{Mass:} \SI{47998438387492700000000}{kg}
\item{Planet density:} \SI{3.013}{g/cm^3}
\item{Surface Area:} \SI{30612893.23}{km^2}
\item{Surface Gravity:} \SI{1.315}{m/s^2}
\item{Escape Velocity:} \SI{7293}{km/h}. Scientific Notation: \SI{2026}{m/s}
\item{Sidereal Rotation Period (Length of Day):} 3.551 Earth days
\item{Atmospheric Constituents:} Oxygen
\item{Temperature:} \SI{102}{K}
\item{Albedo:} 0.67
\item{Solar intensity:} \SI{49.8}{W/m^2}
\end{itemize}

%% D1.2.2. Comparative table including all the relevant design data
%% collected for the reference spacecrafts identified under Task B
%% above.

\deliverable{1.2.2}

Comparative table including all relevant data of reference
spacecrafts.

\begin{longtable}{lp{.8\textwidth}}
  \caption{Reference spacecraft data.} \\ \toprule

  \multicolumn{2}{c}{Launch vehicle} \\ \midrule

  Galileo & Space Shuttle Atlantis (STS-34R) \\

  Juno & Atlas V551 (Atlas first stage with five solid rocket
  boosters, Centaur upper stage) \\

  JEO & \\

  JUICE & \\

  Cassini & Titan IV-B/Centaur launch vehicle \\

  \multicolumn{2}{c}{Mass} \\ \midrule

  Galileo & Orbiter: \SI{2223}{kg} Probe: \SI{339}{kg} \\

  Juno & \SI{3625}{kg} total at launch, consisting of \SI{1593}{kg} of
  spacecraft, \SI{1280}{kg} of fuel and \SI{752}{kg} of oxidizer \\

  JEO & Launch Mass Capability, \SI{5040}{kg} Launch Vehicle Adapter,
  \SI{123}{kg} Flight System, \SI{1367}{kg} Propellant (for
  \SI{2260}{m/s}), \SI{2646}{kg} Remaining usable launch mass,
  \SI{973}{kg} (for contingency and system margin) \\

  JUICE & Dry mass at launch: ~\SI{1800}{kg}. Chemical propellant:
  \SI{3000}{kg}. Payload mass: \SI{104}{kg} \\

  Cassini & Spacecraft \SI{2442}{kg}. Propellant \SI{3132}{kg}. Total
  Mass \SI{5574}{kg}. Orbiter Weight: \SI{5712}{kg} with fuel, Huygens
  probe, adapter, etc. \SI{2125}{kg} unfueled orbiter alone \\

  \multicolumn{2}{c}{Dimensions} \\ \midrule

  Galileo & Orbiter: 6.15 meters Probe: 86 cm \\

  Juno & 3.5 meters high, 3.5 meters in diameter \\

  JEO & \\

  JUICE & \\

  Cassini & Orbiter dimension: 6.7 meters high; 4 meters wide \\

  \multicolumn{2}{c}{Power} \\ \midrule

  Galileo & Two Radio\-isotope Thermoelectric Generators. Max power:
  \SI{570}{W} \\

  Juno & Solar Arrays: length of each solar array \SI{9}{m} by
  \SI{2.65}{m} Total surface area of solar arrays: more than
  \SI{60}{m^2} \\

  JEO & \SI{540}{W} RPS (MMRTG or ASRG) Lithium Ion battery for peak
  power management \\

  Juice & Power: solar panels: \SI{636}{W}--\SI{693}{W} (EOM). Solar array
  \SI{60}{m^2}--\SI{75}{m^2} \\

  Cassini & 885 watts (633 watts at end of mission) from radioisotope
  thermoelectric generators \\

  \multicolumn{2}{c}{Propulsion} \\ \midrule

  Galileo & The spacecraft's propulsion module consists of twelve
  10-newton (2.25 pound\--force) thrusters and a single 400-newton
  (90-pound-force) engine which use monomethylhydrazine fuel and
  nitrogen-tetroxide oxidizer \\

  Juno & Dual mode propulsion subsystem; a bi-propellant main engine and
  mono-propellant reaction control system thrusters. The 12 reaction
  control system thrusters allow translation and rotation about three
  axes \\

  JEO & Bi-propellant , \SI{900}{N} gimbaled engine \\

  JUICE & High delta-V capability (2700 m/s) \\

  Cassini & Two engines, 445 Newton thrust each \\

  \multicolumn{2}{c}{Antenna diameter} \\ \midrule

  Galileo & 4.8-meter \\

  Juno & 4 meter \\

  JEO & \\

  JIME & 3m \\

  Cassini & 4m \\

  \multicolumn{2}{c}{Total cost} \\ \midrule

  Galileo & \$1.3 billion \\

  Juno & \$1.1 billion \\

  JEO & \\

  JIME & \\

  Cassini & \$3.27 billion \\

  \multicolumn{2}{c}{Additional characteristics} \\ \midrule

  Galileo & \\

  Juno & \\

  JEO & \SI{3}{m} HGA with 2-axis gimbal, \SI{25}{W} X-band and
  Ka-band TWTAs \\

  JIME & X and Ka bands, Downlink 1.4 Gbit/day \\

  Cassini & \\ \bottomrule
\end{longtable}

Sources used
\cite{Galileo,Galileo2,Juno,JEO,Juice,Juice2,Cassini,Cassini2}.

%% D1.2.3. Detailed list of design requirements, with a clear
%% indication of the driving requirements.

\deliverable{1.2.3}

Design Requirements
\begin{itemize}

\item{Propulsion}
  \begin{itemize}
  \item{The spacecraft should be able to generate enough thrust in
    order to get in the flight path to Jupiter after being separated
    from the launcher}
  \item{The propulsion should be provided in such a way that direction
    of the flight path can be controlled}
  \item{The spacecraft should be able to provide propulsion which can
    create the desired $\Delta V$ in order to achieve interplanetary
    transfer (Earth -- Jupiter \& Jupiter -- Europa)}
  \item{The propulsion should be provided in case of undesired
    de-orbiting}
  \end{itemize}

\item{Power}
  \begin{itemize}
  \item{At least 50 watts of power should be provided for the payload
    (spacecraft instruments), which will be gained from the solar
    arrays (during exposure to sunlight) and batteries (during
    eclipse)}
  \end{itemize}

\item{Structural Performance}
  \begin{itemize}
  \item{The spacecraft structure should be able to endure the loads
    that are expected to occur during the launch}
  \item{The spacecraft structure should be able to endure the pressure
    difference that is expected to occur during the missions}
  \item{Natural frequency of the spacecraft should be avoided during
    the missions}
  \end{itemize}

\item{Endurance and structural integrity}
  \begin{itemize}
  \item{The spacecraft should resist fatigue and environmental
    deterioration at least over 3 years}
  \item{Reliability should be higher than 0.3}
  \end{itemize}

\item{Data Collection}
  \begin{itemize}
  \item{The spacecraft payload should be able to fully deploy the
    implemented instruments, such as high-resolution camera, laser
    altimeter, ice-penetrating radar, spectrometers, and thermal
    camera}
  \item{Data should be collected and stored securely in the spacecraft
    until the transmission to the ground station}
  \end{itemize}

\item{Communication}
  \begin{itemize}
  \item{Transmission rate should be high enough so that stored data
    can be transmitted properly without being overloaded and/or lost:
    the antenna should have gain that is high enough, and diameter
    that is large enough}
  \end{itemize}

\item{Attitude control and stability}
  \begin{itemize}
  \item{The spacecraft should be able to rotate around all 3-axes to
    control its attitude and point to desired direction}
  \end{itemize}

\item{Thermal control}
  \begin{itemize}
  \item{The spacecraft needs to have required absorptivity and
    emissivity level to achieve a heat balance at
    \SI{150}{K}--\SI{200}{K}}
  \end{itemize}

\end{itemize}

Ten driving requirements:
\begin{enumerate}
\item{The spacecraft should be able to generate enough thrust in order
  to get on the flight path to Jupiter after being separated from the
  launcher:} This is one of the factors that determine the amount of
  fuel and type of main engine/thruster for the spacecraft.
\item{The propulsion should be provided in such a way that the
  direction of the flight path can be controlled:} This
  (maneuverability)has to be taken into account when considering the
  location of the thruster(s).
\item{The spacecraft should have sufficient propulsion in order to
  provide the desired delta-V necessary to achieve interplanetary
  transfer (Earth -- Jupiter \& Jupiter -- Europa):} This should be
  considered when determining the propellant mass.
\item{At least \SI{50}{W} of power should be provided for the payload
  (spacecraft instruments), which will be delivered by the solar
  arrays during exposure to sunlight and batteries during eclipse:}
  This should be considered when determining the size and type of the
  solar array as well as the batteries.
\item{The spacecraft structure should be able to endure the pressure
  difference that is expected to occur during the mission:} This will
  determine the load-bearing structure of the spacecraft.
\item{Reliability should be higher than 0.9:} The reliability of
  subsystems should be considered and selected accordingly.
\item{The spacecraft payload should be able to fully deploy the
  included instruments, such as the high-resolution camera, laser
  altimeter, ice-penetrating radar, spectrometers, and the thermal
  camera:} This will determine the design of the payload and its
  relative location within the spacecraft.
\item{The transmission rate should be high enough so that stored data
  can be transmitted properly without being overloaded and/or lost:
  the antenna should have gain that is high enough, and diameter that
  is large enough:} This should be considered when determining the
  size and type of the antenna.
\item{The spacecraft should be able to rotate in all 3-axes to control
  its attitude and point to the desired direction:} This has to be
  taken into account when considering the location of the thruster(s).
\item{The spacecraft needs to have a heat balance of
  \SI{150}{K}--\SI{200}{K} and therefore requires the correct level of
  absorptivity and emissivity:} This should be considered when
  selecting the outer skin of the spacecraft.

  Sources used \cite{projectreader,fortescue2011spacecraft}.

\end{enumerate}
