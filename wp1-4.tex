
\subworkpkg{1.4}

%% Tasks

%% A. Using methods based on statistical data and/or empirical
%% relationships (like, for instance, the ones presented in Refs. [1]
%% and [2]), define the preliminary mass and power budgets for the
%% spacecraft, i.e.  the mass and power distribution among all the
%% relevant subsystems. Pay particular attention to the definition of
%% proper design margins in your budgets.

\task{A}
Data was obtained from tables from
\cite[p. 589,590]{brown2002elements} for planetary spacecraft and
previous tasks.

The total on-orbit dry mass is

\[ m = 1100\mathrm{kg} \]

The minimum mass contingency for this class of spacecraft and stage of
design 30\%. This works out to a total on-orbit dry mass of 950kg.

Subsystem on-orbit drymass as percentages of total and absolute with
contingency margin removed are the following

\begin{center}
\begin{tabular}{rl}
Structure & $26\% \approx 290\mathrm{kg} - 86\mathrm{kg}$ \\
Thermal & $3\% \approx 33\mathrm{kg} - 9.9\mathrm{kg}$ \\
ACS & $9\% \approx 99\mathrm{kg} - 30\mathrm{kg}$ \\
Power & $19\% \approx 210\mathrm{kg} - 63\mathrm{kg}$ \\
Cabling & $7\% \approx 77\mathrm{kg} - 23\mathrm{kg}$ \\
Propulsion & $13\% \approx 140\mathrm{kg} - 43\mathrm{kg}$ \\
Telecom & $6\% \approx 66\mathrm{kg} - 20\mathrm{kg}$ \\
CDS & $6\% \approx 66\mathrm{kg} - 20\mathrm{kg}$ \\
Payload & $11\% \approx 120\mathrm{kg} - 36\mathrm{kg}$ \\
\end{tabular}
\end{center}

Total power usage is $P \approx 485{W}$.

The minimum power contingency is 90\%. Subsystem power allocation
estimates as percentages of subsystem total ($P_\mathrm{subsystems} =
P - P_\mathrm{payload} \approx 435\mathrm{W}$) and absolute with
contingency margin removed

\begin{center}
\begin{tabular}{rl}
Thermal control & $28\% \approx 120\mathrm{W} - 110\mathrm{W}$ \\
Attitude control & $20\% \approx 87\mathrm{W} - 78\mathrm{W}$ \\
Power & $10\% \approx 44\mathrm{W} - 39\mathrm{W}$ \\
CDS & $17\% \approx 74\mathrm{W} - 67\mathrm{W}$ \\
Communications & $23\% \approx 100\mathrm{W} - 90\mathrm{W}$ \\
Propulsion & $1\% \approx 4.4\mathrm{W} - 3.9\mathrm{W}$ \\
Mechanisms & $1\% \approx 4.4\mathrm{W} - 3.9\mathrm{W}$ \\
\end{tabular}
\end{center}

The launch vehicle adapter mass is given by

\[ M_{\mathrm{adapter}} = 0.0755 M_{\mathrm{launch}} + 50 \]


%% B. Based on the information gathered so far, roughly sketch at
%% least three different preliminary architectures of the spacecraft
%% with its main components (e.g. antennae, solar arrays, batteries,
%% nuclear generator, payload units, external skin). Draft at least
%% one of these preliminary sketches as a CATIA drawing.

%% C. For each one of the proposed preliminary architectures, estimate
%% the vehicle MMOI (Mass Moment of Inertia) both for the deployed and
%% un-deployed state.  Deliverables

%% D1.4.1. Tables showing the preliminary mass and power budgets for
%% the spacecraft.

\deliverable{1.4.1}

%% D1.4.2. Sketches of different optional preliminary spacecraft
%% architectures.

\deliverable{1.4.2}

%% D1.4.3. CATIA drawing of at least one preliminary spacecraft
%% architecture.

\deliverable{1.4.3}

%% D1.4.4. Estimation of the vehicle MMOI in the deployed and
%% un-deployed state.

\deliverable{1.4.4}
