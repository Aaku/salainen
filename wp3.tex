\workpkg{3}

%% This Work Package deals with the detailed sizing and design of an
%% important part of the power subsystem, i.e. the solar arrays.

%% Timeline : Weeks 6 and 7 (2-weeks total duration)

%% Deadline for Report Delivery: Monday October 22nd, 12.00 hours

%% Each report shall include, as a minimum, all the Deliverables
%% listed in the sub-WP descriptions provided below (each deliverable
%% is identified by a unique code, in the form Dx.y.z). The names and
%% student numbers of the authors shall be present on the cover
%% page. The report shall also include a work division table,
%% indicating which group members have contributed to each one of the
%% tasks.  Deliver the printed report to the Teaching Assistant
%% assigned to your group.

\subworkpkg{3.1}

%% Tasks

%% A. Identify at least 5 different types of solar cells that may
%% potentially be used in your spacecraft (hint: look into web sites
%% of solar cell companies like Spectrolab, Emcore, AzurSpace). For
%% each solar cell type, provide its key properties including, as a
%% minimum: power provided in peak power point, short circuit current,
%% open cell voltage, mass, efficiency, solar absorptivity and
%% emissivity, surface area per cell, performance degradation over
%% time, thermal efficiency. You should clearly indicate the
%% conditions (in particular the cell temperature) for which the
%% provided data are valid.

%% B. Make a detailed prediction of the power provided by the solar
%% arrays throughout the spacecraft lifetime. Start as a baseline from
%% the first-order estimation made during WP 2, and refine it by
%% taking into account all relevant additional effects (e.g.: solar
%% cells degradation over time, eclipse periods, variation of the
%% solar cells efficiency with temperature, batteries efficiency,
%% cable losses, possible misalignments of the solar arrays, etc.).

%% C. Define an adequate set of trade-off criteria for the selection
%% of a solar cell type, based on the identified power
%% requirements. Perform the trade-off and select the solar cell type
%% to be used in your spacecraft.

%% D. Based on the actual solar cell type selected and its specific
%% characteristics, define 3 different conceptual designs for the
%% solar arrays configuration of the spacecraft (in terms of geometry,
%% shape, size, number of cells, etc.) and draft them as CATIA
%% drawings.

%% Deliverables

%% D3.1.1. Comparative table of the identified solar cell concepts key
%% properties (including, eventually, pictures and plots).

\deliverable{3.1.1}

%% D3.1.2. Detailed calculations and plots for estimating the power
%% provided by the solar arrays during all mission phases until
%% spacecraft’s EOL.

\deliverable{3.1.2}

%% D3.1.3. Trade-off table(s) for the selection of the solar cell type
%% used in the spacecraft.

\deliverable{3.1.3}

%% D3.1.4. CATIA drawings of 3 different solar arrays configurations
%% for the spacecraft.

\deliverable{3.1.4}

\subworkpkg{3.2}

%% Tasks

%% A. Define a set of trade-off criteria and an appropriate trade-off
%% strategy for selecting one of the 3 different solar arrays
%% configurations; indicate and justify, in particular, the weight to
%% be given to each criterion.

%% B. Analyse the 3 proposed solar arrays configurations in terms of
%% at least the following aspects: total area needed to allocate the
%% required power generation; required shape, dimensions and number of
%% panels; materials, strength and stiffness; design ratios (power to
%% mass, power to area, length to width); mounting on the spacecraft
%% and orientation to the Sun.

%% C. Perform a trade-off of the 3 solar arrays configurations using
%% the criteria and weights defined in Task A above. Justify, also
%% using the results of the analysis performed for the Task B above,
%% ALL the scores given and the final configuration chosen.

%% Deliverables

%% D3.2.1. Detailed analysis of the 3 identified solar arrays
%% configurations with respect to the aspects listed in Task B above.

\deliverable{3.2.1}

%% D3.2.2. Trade-off table(s) for the selection of the solar arrays
%% configuration, including an adequate and clear explanation of all
%% the scores given.

\deliverable{3.2.2}

\subworkpkg{3.3}

%% Tasks

%% A. Determine the size and materials of the supporting structure of
%% the solar arrays and the position of the clamps between different
%% panels. Base your design on the necessity of enabling a safe launch
%% of the solar arrays structure, which translates into a requirement
%% for its eigen-frequency to be higher than 75 Hz.

%% B. Design the interfaces of the solar arrays to the rest of the
%% spacecraft. Consider in particular: the electrical interfaces; the
%% mechanisms for orienting the panels towards the Sun; the hold-downs
%% to be used during transport and launch.

%% C. Design and analyse the kinematics of the solar arrays deployment
%% and of their pointing mechanism.

%% D. Make a CATIA kinematic model of the spacecraft solar arrays.

%% Deliverables

%% D3.3.1. Complete and detailed design of: solar arrays structure;
%% interfaces to the rest of the spacecraft; deployment and pointing
%% mechanisms.

\deliverable{3.3.1}

%% D3.3.2. CATIA drawings and kinematic model of the spacecraft solar
%% arrays.

\deliverable{3.3.2}

\subworkpkg{3.4}

%% Tasks

%% A. Briefly describe the integration of the main and secondary
%% structure items of the solar arrays to the spacecraft, including
%% deployment and folding of the panel stack.

%% B. Briefly discuss the mechanical effects of the solar arrays
%% deployment process on the attitude of the spacecraft.

%% C. Describe the effects of the deployed status of the solar arrays
%% on the attitude of the spacecraft in its final orbit in relation to
%% the relevant disturbance forces as drag, solar wind, radiation
%% pressure.

%% Deliverables

%% D3.4.1. Outcomes of Tasks A, B and C above, including preliminary
%% calculations whereas necessary.

\deliverable{3.4.1}
